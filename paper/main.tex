\documentclass[12pt,a4paper]{report}

% 中文支持
\usepackage{ctex}

% 页面设置
\usepackage[top=2.5cm, bottom=2.5cm, left=2.5cm, right=2.5cm]{geometry}

% 数学公式
\usepackage{amsmath, amssymb, amsfonts}

% 图片
\usepackage{graphicx}
\graphicspath{{figures/}}  % 图片目录

% 表格
\usepackage{booktabs}  % 三线表
\usepackage{multirow}
\usepackage{array}
\usepackage{longtable}

% 代码
\usepackage{listings}
\usepackage{xcolor}

% 超链接
\usepackage[colorlinks=true, linkcolor=blue, citecolor=blue, urlcolor=blue]{hyperref}

% 参考文献(可选)
% \usepackage{cite}

% 算法(可选)
% \usepackage{algorithm}
% \usepackage{algorithmic}

% 代码高亮设置
\lstset{
    basicstyle=\ttfamily\footnotesize,
    keywordstyle=\color{blue},
    commentstyle=\color{gray},
    stringstyle=\color{red},
    breaklines=true,
    frame=single,
    numbers=left,
    numberstyle=\tiny,
    tabsize=4
}

% 标题设置
\usepackage{titlesec}

% 论文信息
\title{\textbf{面向小程序隐私合规检测的自动化框架}\\
\textbf{MiniEval 设计与研究}}
\author{作者姓名}
\date{\today}

\begin{document}

% 封面
\maketitle

% 中文摘要
\begin{abstract}
随着移动互联网的快速发展,小程序已成为用户日常生活中的重要应用形态。然而,大量小程序在收集用户个人信息时存在隐私声明不规范的问题,主要表现为必要性违规和表述模糊违规。针对这一问题,本文提出了MiniEval框架,通过结合静态程序分析和大语言模型技术,实现隐私合规的自动化检测。本文重点研究了隐私声明的合理性分类问题,提出了基于深度提示工程的分类方法,并通过实验验证了其有效性。

\textbf{关键词:} 隐私合规;小程序;大语言模型;提示工程;自动化检测
\end{abstract}

% 目录
\tableofcontents

% 图表目录
\listoffigures
\listoftables

% 第一章
\chapter{绪论}
\label{chap:intro}
本章内容待补充...

% 第二章
\chapter{MiniEval 框架设计与行为违规检测}
\label{chap:mineval}
本章内容待补充...

% 第三章(核心章节)
\chapter{隐私声明目的合理性分类}
\label{chap:classification}

\section{引言}

随着移动互联网的快速发展,小程序已成为用户日常生活中的重要应用形态。然而,大量小程序在收集用户个人信息时存在隐私声明不规范的问题,主要表现为两大类违规:
\begin{enumerate}
    \item \textbf{必要性违规}:数据收集目的与手段之间存在逻辑矛盾,或存在隐私侵害更小的替代方案
    \item \textbf{表述模糊违规}:声明文本表述不清、场景缺失或过于宽泛,导致用户无法准确理解数据用途
\end{enumerate}

针对上述问题,本章提出了一种基于大语言模型的隐私声明合理性分类方法,旨在自动化识别这两类违规声明。

\section{问题定义与挑战}

\subsection{任务定义}

给定一条隐私声明文本 $s$,分类任务的目标是判断其在两个维度上的违规情况:
\begin{itemize}
    \item 必要性违规:$y_{\text{nec}} \in \{0, 1\}$,其中 $0$ 表示合理,$1$ 表示存在必要性违规
    \item 表述模糊违规:$y_{\text{amb}} \in \{0, 1\}$,其中 $0$ 表示清晰,$1$ 表示存在表述模糊违规
\end{itemize}

\subsection{技术挑战}

\textbf{挑战1:} 隐私合规涉及法律、技术、业务等多领域知识,通用模型缺乏隐性领域知识。

\textbf{挑战2:} 违规判定存在模糊边界,例如"为了提升服务质量"这类表述在某些场景下可接受,在其他场景下则构成违规。

\textbf{挑战3:} 在线LLM推理成本高昂,难以大规模部署。

\section{方法}

本章提出三阶段技术路线,如图\ref{fig:pipeline}所示。

\begin{figure}[htbp]
    \centering
    \includegraphics[width=0.9\textwidth]{figures/pipeline.pdf}
    \caption{隐私声明分类三阶段技术路线}
    \label{fig:pipeline}
\end{figure}

\subsection{阶段一:深度提示工程}

\subsubsection{分类提示词设计}

针对必要性违规和表述模糊违规,分别设计专门的提示词。

\textbf{必要性违规提示词:}
包含目的异常和可替代冗余的详细判断准则。

\textbf{表述模糊违规提示词:}
包含语义重复、场景缺失、表述宽泛的判断准则。

\subsubsection{负面锚点与对比示例}

引入负面锚点(Negative Anchors)帮助模型界定违规边界:
\begin{itemize}
    \item 目的异常示例:"为了系统开发,开发者收集你的位置信息"
    \item 可替代冗余示例:"为了上传头像而申请相册写入权限"
    \item 场景缺失示例:"为了上传图片,开发者需要获取你选中的图片"
\end{itemize}

\subsubsection{思维链引导}

强制模型先生成判别依据,再输出标签:
\begin{verbatim}
思考步骤:
1. 分析声明中的数据类型
2. 判断收集目的是否合理
3. 检查是否存在更小侵害的替代方案
4. 给出最终判定
\end{verbatim}

\subsection{阶段二:指令微调(未来工作)}

\subsubsection{指令数据构建}

基于阶段一的高质量预测结果,构建指令微调数据集:
$$\mathcal{D} = \{(\text{instruction}, \text{input}, \text{output})_i\}_{i=1}^N$$

\subsubsection{参数高效微调}

采用LoRA/QLoRA技术对Llama-3-8B等开源基座进行微调。

\subsection{阶段三:模型蒸馏(未来工作)}

使用知识蒸馏技术将大模型能力迁移到1.5B/3B参数量的学生模型,实现工程化部署。

\section{实验}

\subsection{数据集}

\subsubsection{数据收集}

从真实小程序中收集隐私声明,最终构建包含1137条标注样本的数据集。

\subsubsection{标注维度}

每条声明在两个维度上进行标注:必要性违规和表述模糊违规。

\subsubsection{数据分布}

表\ref{tab:data_distribution}展示了数据集的标签分布情况。

\begin{table}[htbp]
    \centering
    \caption{数据集标签分布}
    \label{tab:data_distribution}
    \begin{tabular}{lccc}
        \toprule
        维度 & 负样本(正常/清晰) & 正样本(违规/模糊) & 总计 \\
        \midrule
        必要性违规 & 910 (80.0\%) & 227 (20.0\%) & 1137 \\
        表述模糊违规 & 608 (53.5\%) & 529 (46.5\%) & 1137 \\
        \bottomrule
    \end{tabular}
\end{table}

\subsection{实验设置}

\subsubsection{基线模型}

\textbf{BERT系列:}
\begin{itemize}
    \item RoBERTa-wwm-ext多标签分类
    \item RoBERTa-wwm-ext单任务分类(必要性)
    \item RoBERTa-wwm-ext单任务分类(表述模糊)
\end{itemize}

\textbf{训练配置:}
\begin{itemize}
    \item 优化器:AdamW
    \item 学习率:$2 \times 10^{-5}$
    \item 批次大小:16
    \item 训练轮数:最多10轮(带早停,Patience=3)
    \item 学习率调度:余弦退火
    \item 类别权重:Balanced
\end{itemize}

\subsubsection{LLM配置}

\textbf{模型:} Qwen-Plus(阿里云DashScope API)

\textbf{提示词:}
\begin{itemize}
    \item 统一分类提示词(同时评估两个维度)
    \item 必要性独立提示词
    \item 表述模糊独立提示词
\end{itemize}

\subsubsection{评估指标}

\begin{itemize}
    \item 准确率(Accuracy)
    \item 精确率(Precision)
    \item 召回率(Recall)
    \item F1分数(F1-Score)
\end{itemize}

\subsection{实验结果}

\subsubsection{基线模型性能}

表\ref{tab:baseline_results}展示了BERT系列模型的性能。

\begin{table}[htbp]
    \centering
    \caption{BERT基线模型性能}
    \label{tab:baseline_results}
    \begin{tabular}{lcccc}
        \toprule
        模型 & Accuracy & Precision & Recall & F1 \\
        \midrule
        多标签分类 & 待填入 & 待填入 & 待填入 & 待填入 \\
        必要性(单任务) & 待填入 & 待填入 & 待填入 & 待填入 \\
        表述模糊(单任务) & 待填入 & 待填入 & 待填入 & 待填入 \\
        \bottomrule
    \end{tabular}
\end{table}

\subsubsection{LLM分类器性能}

表\ref{tab:llm_results}展示了LLM分类器的性能。

\begin{table}[htbp]
    \centering
    \caption{LLM分类器性能}
    \label{tab:llm_results}
    \begin{tabular}{lcccc}
        \toprule
        分类器 & Accuracy & Precision & Recall & F1 \\
        \midrule
        统一分类 & 待填入 & 待填入 & 待填入 & 待填入 \\
        必要性(独立) & 待填入 & 待填入 & 待填入 & 待填入 \\
        表述模糊(独立) & 待填入 & 待填入 & 待填入 & 待填入 \\
        \bottomrule
    \end{tabular}
\end{table}

\subsubsection{对比分析}

图\ref{fig:comparison}展示了BERT基线与LLM分类器的性能对比。

\begin{figure}[htbp]
    \centering
    \includegraphics[width=0.8\textwidth]{figures/comparison.pdf}
    \caption{BERT vs LLM性能对比}
    \label{fig:comparison}
\end{figure}

\subsubsection{一致性分析}

表\ref{tab:consistency}展示了统一分类与独立分类的一致性分析。

\begin{table}[htbp]
    \centering
    \caption{分类器一致性分析}
    \label{tab:consistency}
    \begin{tabular}{lcc}
        \toprule
        维度 & 一致率 & Cohen's Kappa \\
        \midrule
        必要性 & 待填入 & 待填入 \\
        表述模糊 & 待填入 & 待填入 \\
        \bottomrule
    \end{tabular}
\end{table}

\subsection{消融实验}

\subsubsection{提示策略影响}

表\ref{tab:ablation}展示了不同提示策略对性能的影响。

\begin{table}[htbp]
    \centering
    \caption{提示策略消融实验}
    \label{tab:ablation}
    \begin{tabular}{lcc}
        \toprule
        提示策略 & 必要性 F1 & 表述模糊 F1 \\
        \midrule
        基础提示 & 待填入 & 待填入 \\
        + 锚点示例 & 待填入 & 待填入 \\
        + 思维链 & 待填入 & 待填入 \\
        + 对比示例 & 待填入 & 待填入 \\
        \bottomrule
    \end{tabular}
\end{table}

\subsection{案例分析}

\subsubsection{正确案例}

\textbf{示例1:} "为了门店打卡和位置校准,开发者收集你的位置信息"

\textit{分析:} 目的明确且必要,两个模型均判定为正常。

\subsubsection{错误案例}

\textbf{示例2:} "为了系统开发,开发者收集你的��置信息"

\textit{分析:} BERT误判为正常,LLM正确识别为目的异常。

\section{讨论}

\subsection{单任务 vs 多任务分类}

实验表明,单任务分类在两个维度上均优于多任务分类,可能原因:
\begin{itemize}
    \item 两个任务的决策逻辑存在差异
    \item 单任务可以针对特定维度优化类别权重
\end{itemize}

\subsection{统一分类 vs 独立分类}

统一分类在效率上更优(一次调用),但独立分类在性能上略优,且可解释性更强。

\subsection{模型选择建议}

\begin{itemize}
    \item \textbf{高精度场景}:使用LLM独立分类器
    \item \textbf{大规模检测}:使用BERT单任务模型
    \item \textbf{实时推理}:考虑使用蒸馏后的轻量模型
\end{itemize}

\section{本章小结}

本章针对隐私声明合理性分类问题,提出了基于深度提示工程的LLM分类方法。通过设计专门的提示词、引入负面锚点和对比示例、结合思维链引导,显著提升了分类性能。实验结果表明:

\begin{enumerate}
    \item LLM分类器在两个维度上均优于BERT基线
    \item 单任务分类优于多任务分类
    \item 深度提示工程能有效激发模型性能
\end{enumerate}

未来工作将进一步探索指令微调和模型蒸馏技术,实现性能与效率的平衡。


% 第四章
\chapter{隐私风险量化评估与生态实证分析}
\label{chap:risk}
本章内容待补充...

% 第五章
\chapter{总结与展望}
\label{chap:conclusion}
本章内容待补充...

% 参考文献(如果需要)
% \bibliographystyle{plain}
% \bibliography{references}

% 附录(可选)
% \appendix
% \chapter{补充材料}

\end{document}
